\documentclass[a4paper, 12pt]{article}
\usepackage[ngerman]{babel}
\usepackage[utf8]{inputenc}
\usepackage{geometry}
% for hyperlinks
\usepackage{hyperref}
% for some nice pics
\usepackage{graphicx}
% for algorithm env
\usepackage{algpseudocode, algorithm}
% \usepackage[pdftex]{graphicx}


\geometry{a4paper,left=25mm,right=25mm, top=2cm, bottom=2cm}

\begin{document}
% Title Page
\input{./title_page_1.tex}
% Table of Contents
\tableofcontents

\begin{abstract}
	This ist my abstract.
\end{abstract}

\section{LeitfragenTEMP}
\begin{itemize}
\item Was sind opportunistische Netze (OppNets) und welche Untertypen gibt es?
\item Wo liegen die besonderen Herausforderungen bei OppNets?
\item In welchen Anwendungsgebieten finden OppNets Verwendung?
\item Welche Arten von Forwarding-Protokollen gibt es fuer OppNets und worin unterscheiden sie sich?
\item Welche Modelle sind im Kontext der Leistungsbewertung von OppNets besonders relevant und worin liegen deren Vor- und Nachteile?
\item Was sind aktuelle und offene Probleme/Fragestellungen?
\end{itemize}

\section{Einleitung}
Welche Fragestellungen werden behandelt?
Warum sind diese Fragestellungen interessant? Hierzu eventuell größeren Kontext vorstellen...

Am Ende noch ein Ausblick auf die einzelnen Kaptel, der skizziert, wie die Fragestellung bearbeitet wurde.


\section{Harter Stoff}
Kern der Arbeit mit mehreren Kapiteln(sub). 



\section{Schluss}
Zusammenfassen wesentlicher Asplete der Arbeit (Essenz).
Sie sollte zu verstehen sein, ohne den Hauptteil zu kennen (Bild machen können so).
eigene Meinung über die Fragestellung genau hier.
evtl ein ausblick auf mögliche weiterführende arbeiten???


\end{document}